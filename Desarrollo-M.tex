\documentclass[11pt,letterpaper]{article}
\usepackage[utf8]{inputenc}
\usepackage[spanish]{babel}
\usepackage{amsmath}
\usepackage{amsfonts}
\usepackage{amssymb}
\usepackage{float}
\usepackage{graphicx}
\usepackage{subfigure}
\usepackage{listings}
\usepackage{verbatim}
\usepackage[left=2cm,right=2cm,top=2cm,bottom=2cm]{geometry}
\begin{document}
\title{\huge{\textbf{ Desarrolo matematico}}}
\author{\small \textit{M. Sabogal $^{1}$}\\
	\small \textit{ $^{1}$ Programa de Física, Universidad del Atlántico, Km 7 vía a Puerto Colombia, Barranquilla, Colombia}} %Código Postal 081001
\date{} 


Ecuacion para el énesimo resonador acoplado: 
\begin{equation}
\dfrac{d^{2}}{dt^{2}} \left[ q_{n} + \lambda \left( q_{n+1} + q_{n-1}\right) \right] = - \omega_{o}^{2} q_{n} + \chi \omega_{o}^{6} q_{n}^{3} 
\end{equation}
\par 
Para $n=1,2,3,4 \cdots n$; donde $q_{n+1}=0$, se tiene el sistema en el lado derecho de la ecuacion $1$:
$$ \dfrac{d^{2}}{dt^{2}} \left(
\begin{matrix}
q_{1} & \lambda q_{2} & 0 & 0 & 0 & \cdots & 0 & 0 \\
\lambda q_{1} & q_{2} & \lambda q_{3} & 0 & 0 & \cdots & 0 & 0\\
0 & \lambda q_{2} & q_{3} & \lambda q_{4}  & 0 & \cdots & 0& 0 \\
0 & 0 & \lambda q_{3} & q_{4} & \lambda q_{5}  & \cdots & 0 & 0 \\
\vdots & \vdots & \vdots & \ddots & \ddots & \ddots & \vdots & \vdots \\
0 & 0 & 0 & \cdots & \lambda q_{n-3} &  q_{n-2} & \lambda q_{n-1} &0\\    
0 & 0 & 0 & \cdots & 0 & \lambda q_{n-2} & q_{n-1} & \lambda q_{n}\\    
0 & 0 & 0 & \cdots & 0 & 0 & \lambda q_{n-1} & q_{n}     
\end{matrix} \right) =  \dfrac{d^{2}}{dt^{2}} \left[ 
\left(
\begin{matrix}
1 & \lambda  & 0 & 0 & 0 & \cdots & 0 & 0 \\
\lambda & 1 & \lambda  & 0 & 0 & \cdots & 0 & 0\\
0 & \lambda  & 1 & \lambda   & 0 & \cdots & 0& 0 \\
0 & 0 & \lambda  & 1 & \lambda   & \cdots & 0 & 0 \\
\vdots & \vdots & \vdots & \ddots & \ddots & \ddots & \vdots & \vdots \\
0 & 0 & 0 & \cdots & \lambda  &  1 & \lambda &0\\    
0 & 0 & 0 & \cdots & 0 & \lambda  & 1 & \lambda\\    
0 & 0 & 0 & \cdots & 0 & 0 & \lambda  & 1    
\end{matrix} \right) \left(
\begin{matrix}
q_{1}\\
q_{2}\\
q_{3}\\
q_{4}\\
\vdots\\
q_{n-3}\\
q_{n-2}\\
q_{n-1}\\
q_{n}
\end{matrix} \right) \right] $$
\par 
Reescribiendo de la forma: 
\begin{equation}
\left(
\begin{matrix}
1 & \lambda  & 0 & 0 & 0 & \cdots & 0 & 0 \\
\lambda & 1 & \lambda  & 0 & 0 & \cdots & 0 & 0\\
0 & \lambda  & 1 & \lambda   & 0 & \cdots & 0& 0 \\
0 & 0 & \lambda  & 1 & \lambda   & \cdots & 0 & 0 \\
\vdots & \vdots & \vdots & \ddots & \ddots & \ddots & \vdots & \vdots \\
0 & 0 & 0 & \cdots & \lambda  &  1 & \lambda &0\\    
0 & 0 & 0 & \cdots & 0 & \lambda  & 1 & \lambda\\    
0 & 0 & 0 & \cdots & 0 & 0 & \lambda  & 1    
\end{matrix} \right) \dfrac{d^2}{dt^2} \left(
\begin{matrix}
q_{1}\\
q_{2}\\
q_{3}\\
q_{4}\\
\vdots\\
q_{n-3}\\
q_{n-2}\\
q_{n-1}\\
q_{n}
\end{matrix} \right)  = \hat{M_{1}} \dfrac{d^{2} \vec{q}_{n}}{dt^{2}}
\end{equation}
\par 
De manera similar para el lado derecho de $1$, se tiene el sistema:  
$$ \left(
\begin{matrix}
- \omega_{o}^{2} q_{1} + \chi \omega_{o}^{6} q_{1}^{3} & 0 & 0 & 0 & \cdots & 0 \\
0  & - \omega_{o}^{2} q_{2} + \chi \omega_{o}^{6} q_{2}^{3} & 0 & 0  & \cdots & 0 \\
0 & 0 & - \omega_{o}^{2} q_{3} + \chi \omega_{o}^{6} q_{3}^{3} & 0 & \cdots & 0 \\
0 & 0 & 0 & - \omega_{o}^{2} q_{4} + \chi \omega_{o}^{6} q_{4}^{3} & \cdots & 0 \\
\vdots & \vdots & \vdots & \vdots & \ddots & \vdots & \\
0 & 0 & 0 & 0& \cdots &- \omega_{o}^{2} q_{n} + \chi \omega_{o}^{6} q_{n}^{3}\\     
\end{matrix} \right) = $$ 

\begin{equation}
\left(
\begin{matrix}
- \omega_{o}^{2} + \chi \omega_{o}^{6} q_{1}^{2} & 0 & 0 & 0 & \cdots & 0 \\
0  & - \omega_{o}^{2} + \chi \omega_{o}^{6} q_{2}^{2} & 0 & 0  & \cdots & 0 \\
0 & 0 & - \omega_{o}^{2} + \chi \omega_{o}^{6} q_{3}^{2} & 0 & \cdots & 0 \\
0 & 0 & 0 & - \omega_{o}^{2} + \chi \omega_{o}^{6} q_{4}^{2} & \cdots & 0 \\
\vdots & \vdots & \vdots & \vdots & \ddots & \vdots & \\
0 & 0 & 0 & 0& \cdots &- \omega_{o}^{2} + \chi \omega_{o}^{6} q_{n}^{2}\\     
\end{matrix} \right) \left( \begin{matrix}
q_{1}\\
q_{2}\\
q_{3}\\
q_{4}\\
\vdots\\
q_{n}
\end{matrix} \right) = \hat{M_{2}} \vec{q}_{n}
\end{equation}
\par 
Igualando $2$ y $3$ se tiene : 
\begin{equation}
\hat{M_{1}} \dfrac{d^{2} \vec{q}_{n}}{dt^{2}} = \hat{M_{2}} \vec{q}_{n} \hspace*{0.3 cm} \rightarrow \hspace*{0.3 cm}\dfrac{d^{2} \vec{q}_{n}}{dt^{2}} = \hat{M_{1}^{-1}} \hat{M_{2}} \vec{q}_{n} =  \hat{M_{3}} \vec{q}_{n}
\end{equation}
\par 
Donde $\hat{M_{3}}=\hat{M_{1}^{-1}} \hat{M_{2}}$. 
\end{document}
